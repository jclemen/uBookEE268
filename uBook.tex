\documentclass[12pt]{report}
\usepackage{textgreek}
\usepackage{fullpage}
\usepackage[final]{pdfpages}

\begin{document}
\title{Intro to Modern Optics uBook}
\author{James Clements}
\date{Fall 2012}
\maketitle
\tableofcontents
\part{Physics of Light}
\chapter{Wave Motion}
Light acts in some ways as a wave and in other ways like a particle (does that make it a wavicle?). Understanding the basics of waves is useful for studying light. We'll examine these first as one dimensional waves to develop some first principles and then expand these fundamentals to apply to two dimensional and three dimensional waves.
\section{One-Dimensional Waves}
This is the easiest way to consider a wave. Consider perturbing a string or a spring by suddenly jerking one end upward and then back to it's original position. This will cause a perturbation to travel through the object. The actual material does not permanently deform as it returns to its original state and thus the disturbance advances and not the medium. The disturbance of a wave is a function of position and time and is denoted by the symbol \textpsi : 
\[\psi (x,t) = f(x,t)\]

The profile of the wave is determined by setting time equal to zero as in: 
\[\psi(x,t)|_{t=0} = f(x,0) = f(x) \]

When considering time, a wave travels at a specific velocity \emph{v}. The distance a wave travels is simply \emph{vt}. An alternate frame of reference, \emph{S'} can be used which travels along with the pulse in time. In this frame, the pulse always looks identical to the profile when \emph{t=0}. Here, the coordinate is \emph{x'} rather than \emph{x} such that $ \psi = f(x') $. The relationship between \emph{x} and \emph{x'} is: $ x' = x - vt $. To describe the wave as someone would observe from the original reference frame, \emph{S}, we can now write that:
\[\psi(x,t) =f(x-vt)\]
This is the general form of the one-dimensional wavefunction. 
\subsection{The Differential Wave Equation}
The differential wave equation is a linear, homogeneous, second order, partial differential equation that is usually taken as the defining expression for physical waves in a lossless medium. The one-dimensional form of the wave equation is derived from the initial relation ship of $\psi(x,t) = f(x')$. Derivatives are taken twice (see Hecht, 4th edition pages 13-14 for details) to bring to the final result of:
\begin{equation}
\frac{\partial ^2\psi}{\partial x^2} = \frac{1}{v^2}\frac{\partial ^2\psi}{\partial t^2}
\end{equation}
This is the wave equation for undamped systems that do not contain sources in the region under consideration. Damping effects would be considered using a $\frac{\partial \psi}{\partial t} $ term.

\section{Harmonic Waves}
Harmonic waves have a sinusoidal profile. Any wave shape can be synthesized by a superposition of harmonic waves, so they're pretty useful. The simplest profile of a harmonic wave can be expressed as: \[\psi(x,t)|_{t=0}=\psi(x) = A\sin kx = f(x)\] where \emph{A} is the amplitude of the wave and \emph{k} is a positive constant known as teh propagation number. Transforming this into a progressive wave yields: 
\begin{equation}
\psi(x,t) = A\sin k(x-vt) = f(x-vt)
\end{equation} 
The symbol $\lambda$ represents the wavelength (also known as spatial period) of the wave and is related to $k$ by the following equation: 
\begin{equation}
k = 2\pi /\lambda
\end{equation}

The amount of time it takes for one complete wave to pass a stationary observer is defined as the temporal period, \texttau . Propagation number, wave velocity, and temporal period are related by the following relationship: \[kv\tau=2\pi\] it also follows that \[\tau=\lambda/v\]

The temporal frequency, \textnu , is the number of waves per unit time (often measured in Hertz) and is related to the above terms under the following equations:
\begin{equation}
\nu \equiv 1/\tau
\end{equation}
\begin{equation}
v = \nu\lambda
\end{equation}

Other related useful terms are the angular temporal frequency, \textomega , and the wave number (spatial frequency), $\kappa$, which are defined respectively as:
\begin{equation}
\omega \equiv 2\pi/\tau = 2\pi\nu
\end{equation}
\begin{equation}
\kappa \equiv 1/\lambda
\end{equation}

Another important note is that no wave is monochromatic, meaning that it has perfect frequency. All waves fall into a band of frequencies. When that band is small, the wave is termed quasimonochromatic. 

\section{Phase and Phase Velocity}
Wave equations are often written in the form: 
\begin{equation}
\psi(x,t) = A \sin (kx-\omega t +\epsilon)
\end{equation}
Wherein the portion inside the sine term consists of the position of the wave, $kx$, the time state of the wave, $\omega t$, and a constant, $\epsilon$ that defines the initial phase of the wave. Without the initial phase, the function would always be zero at the origin of space and time.

Note once again that $\omega$ is the rate of change of phase with time:
\[|(\frac{\partial\psi}{\partial t})_x|=\omega\]
the rate of change of phase with distance keeping t constant is $k$:
\[|(\frac{\partial\psi}{\partial x})_t|=k\]
and the phase velocity, $v$, is the speed at which the wave propagates in space:
\[(\frac{\partial x}{\partial t})_\psi = \pm \frac{\omega}{k} = \pm v  \]

\section{The Superposition Principal (basics)}
Since the differential wave equation is a linear partial differential equation, it holds that the sum of two individual solutions to the wave equation is also a solution to the wave equation. When two separate waves overlap in space, the resulting disturbance at each point in the region of overlap is the algebraic sum of the individual constituent waves at that location. 

Waves are said to be in-phase when their phase angles are identical and can be out of phase to a limit of having a phase angle difference of \textpi. Out of phase waves give rise to interference. 

\section{The Complex Representation}
Euler's formula, $e^{i\theta} = \cos\theta+i\sin\theta$, is often a mathematically optimal way to express harmonic waves since operations such as taking a derivative and multiplying functions is much easier. It is often most convenient to express the harmonic wave as:
\begin{equation}
\psi(x,t) = Ae^{i(\omega t -kx+\epsilon)}
\end{equation}
An important note is that while the imaginary portion of the function is kept out of convenience through calculations, the real part of the equation is the actual expression of the wave. This is only done after obtaining the final result of all calculations.
\section{Phasors and the Addition of Waves}
Phasors are a useful abstraction for understanding waves. Phasor notation contains the amplitude and current phase angle of the wave. Phase angle is the angle by which the wave is offset from its reference state. Phasors are expressed as $$A\angle \phi $$ where $A$ is the maximum amplitude of the wave and $\phi$ is its phase angle. 

When combining wavefunctions, phasors can be used similarly to vectors. The wavefunctions being summed are added head to tail in order to determine the resultant vector. 

\section{Plane Waves}
A plane wave is the simplest example of a three-dimensional wave, but it is extremely useful as all other three-dimensional waves can be described as a combination of plane waves.

Plane waves travel along a propagation vector $\vec{k}$ whose magnitude is the propagation number, k, which has already been described in terms of harmonic waves. The equation of a plane, r, which is perpendicular to $\vec{k}$ is:
\[\vec{k}\cdot\vec{r} = a\] where $a$ is a constant. 

The general equation of a harmonic plane wave in Cartesian coordinates is:
\[\psi(x,y,z,t) = Ae^{i(k_xx+k_yy+k_zz \mp \omega t)}\]

\section{The Three-Dimensional Wave Equation}
The three-dimensional wave equation is extremely similar to the 1-dimensional version. The only difference is that three spatial variables are taken into account. The 3-D wave equation takes on the form:
\begin{equation}
\nabla^2 \psi = \frac{1}{v^2}\frac{\partial\psi}{\partial t^2}
\end{equation}

where $\nabla$ is the Laplacian operator: $\nabla \equiv \frac{\partial ^2}{\partial x^2}+\frac{\partial ^2}{\partial y^2}+\frac{\partial ^2}{\partial z^2}$

\section{Spherical Waves}
Spherical waves represent a point source radiating outward or a spherical shell radiating inward. The harmonic spherical wave equation is given as:
\begin{equation}
\psi (r,t) = \left( \frac{\mathcal{A}}{r} \right)\cos k(r \mp vt )
\end{equation}

\section{Cylindrical Waves}
Cylindrical waves do not have a clean solution to their differential wave equation. Bessel's equation can be used to approximate cylindrical waves of large radii:
\[\psi(r,t) \approx \frac{\mathcal{A}}{\sqrt{r} \cos k(r \mp vt)}\]

This equation best approximates what happens to a plane wave that encounters a long, narrow slit. 

\section{Glossary}
\begin{description}
\item[amplitude:] The maximum disturbance of a wave.
\item[angular temporal frequency:] The number of phase angle changes per unit time, denoted as \textomega .
\item[harmonic waves:] A wave that can be represented using sine or cosine curves. 
\item[in-phase:] Multiple waves having a phase-angle difference of zero are in phase. The disturbance of the waves sums maximally causing a much greater intensity resultant wave.
\item[initial phase (\textepsilon ):] The angle which is the constant contribution to the phase arising at the wave generator. This is independent of how far in space or how long in time the wave has traveled. 
\item[longitudinal wave:] A wave in which the  medium is displaced in the direction of the motion of the wave.
\item[monochromatic:] A wave which travels at constant frequency.
\item[out-of-phase:] Multiple waves having a phase-angle difference of 180$^{\circ}$ are said to be out of phase. The waves interfere with each other such that the resultant wave disturbance is minimized. 
\item[phase velocity:] The speed at which a wave profile moves. Denoted as $v = \frac{\omega}{k}$.
\item[phasor:] An abstraction useful in expressing a harmonic wave in terms of its amplitude, \emph{A} and phase, $\phi$ as $A \angle \phi$. 
\item[plane wave:] A planar wave that is perpendicular from a direction vector, $\mathbf{\vec{k}}$.
\item[propagation number:] A positive constant denoted as \emph{k} used in studying harmonic waves which ensures correct units inside the sine function and can change the period of the sine wave. When \textlambda  is defined as the wavelength, $k=2\pi /\lambda$.
\item[spacial frequency:] The number of waves per unit length, denoted by $\kappa$. Synonymous with wave number. 
\item[superposition principle:] The principle in which multiple waves traveling along the same path are summed. 
\item[temporal frequency:] The number of waves per unit time, denoted as \textnu. Often takes on units of Hertz (Hz). 
\item[temporal period:] Denoted as \texttau. This is the amount of time it takes for one complete wave to pass a stationary observer.
\item[transverse wave:] A wave in which the medium is displaced in a direction perpendicular to that of the motion of the wave
\item[traveling wave:] A wave whose crest travels across particles.
\item[wavefront:] The surfaces of a three-dimensional wave that join all points of equal phase. 
\item[wave number:] The number of waves per unit length, denoted by $\kappa$. Synonymous with spacial frequency. 
\end{description}
\section{Important Equations}
\[\frac{\partial ^2\psi}{\partial x^2} = \frac{1}{v^2}\frac{\partial ^2\psi}{\partial t^2}\]
\[\psi(x,t) = A\sin k(x-vt) = f(x-vt)\]
\[k = 2\pi /\lambda\]
\[\nu \equiv 1/\tau\]
\[v = \nu\lambda\]
\[\omega \equiv 2\pi/\tau = 2\pi\nu\]
\[\kappa \equiv 1/\lambda\]
\[\psi(x,t) = A \sin (kx-\omega t +\epsilon)\]
\[\psi(x,t) = Ae^{i(\omega t -kx+\epsilon)}\]
\[\nabla^2 \psi = \frac{1}{v^2}\frac{\partial\psi}{\partial t^2}\]
\[\psi (r,t) = \left( \frac{\mathcal{A}}{r} \right)\cos k(r \mp vt )\]

\section{Homework Problems}

Due October 4, 2012.

Problems from Hecht Optics Chapter 2: numbers: 4, 13, 17, 18, 22, 32
Solutions were hand written and are shown on the following pages. 
\includepdf[ pages=- ]{EE268HW1}


\chapter{Superposition of Waves}
For certain types of waves, specifically small-amplitude linear systems, the Principle of Superposition is able to be used as a convenient method for determining how multiple waves will interact in a system. The  principle of superpositions suggests that the resultant disturbance at any point in a medium is the algebraic sum of the separate constituent waves. 

\section{Addition of Waves of the Same Frequency}

\subsection{The Algebraic Method}

Take two harmonic electromagnetic waves, $E_1$ and $E_2$, of the form \[E(x,t) = E_0 \sin [ \omega t + \alpha(x,\epsilon)]\]  (where $\alpha(x,\epsilon) = -(kx+\epsilon)$ is used to separate the spatial terms from the temporal terms) are occupying the same place in space such that they interact with one another. By grouping terms the equations can be written out as: 
\[E_1 = E_{01} \sin (\omega t + \alpha_1 )\] \[E_2 = E_{02} \sin (\omega t + \alpha_2 )\]

The resulting disturbance, $E = E_1+E_2$ can be written as:
\begin{equation}
E = E_0 \sin (\omega t + \alpha)
\end{equation}
where
\begin{equation}
E_0 ^2 = E_{01} ^2 + E_{02} ^2 + 2E_{01}E_{02} \cos (\alpha_2 - \alpha_1)
\end{equation}
\begin{equation}
\tan \alpha = \frac{E_{01} \sin \alpha_1 + E_{02} \sin \alpha_2}{E_{01} \cos \alpha_1 + E_{02} \cos \alpha_2}
\end{equation}

The term: $2E_{01}E_{02} \cos (\alpha_2 - \alpha_1)$ is the interference term which contains the crucial phase difference term: \[\delta \equiv (\alpha_2-\alpha_1)\] which can arise from a difference in path length as well as a difference in initial phase angle such that:
\[\delta = (kx_1+\epsilon_1) - (kx_2+\epsilon_2)\]
\[\delta = \frac{2\pi}{\lambda}(x_1-x_2)+(\epsilon_1-\epsilon_2)\]
where $x_1$ and $x_2$ are distances from the sources of the two waves to the point of observation. If the waves are initially in-phase at their sources, then $\epsilon_1 = \epsilon_2$.

When two disturbances from the same source travel different routes before arriving at the point of observation, \[\delta = \frac{2\pi}{\lambda_0}n(x_1-x_2)\] where $n$ is the index of refraction of the medium. $n(x_1-x_2)$ is known as the optical path difference (OPD or $\Lambda$).

Also, when $\epsilon_1-\epsilon_2$ is a constant value, the waves are said to be coherent. For multiple in-phase, coherent sources: \[E_0 ^2 = \left( \displaystyle\sum\limits_{i=1}^N E_{0i} \right)^2 \] which simplifies to the following when all the amplitudes are the same value of $E_{01}$:
\[E_0 ^2 = N^2 E_{01}^2\] whereas for incoherent light: \[E_0 ^2 = N E_{01}^2\]

\subsection{Standing Waves}

A standing wave consists of 2 harmonic waves (possibly a reflection of a wave from the same source) which have the same frequency and period, but travel in opposite directions. A standing wave, or stationary wave, has a profile that does not move through space. Considering a case with two waves, the incident wave: $E_I$, and the reflected wave: $E_R$:
\[E_I = E_{0I} \sin (kx+\omega t+\epsilon_I)\]
\[E_R = E_{0R} \sin (kx+\omega t+\epsilon_R)\]
The boundary condition of the standing wave requires that $\epsilon_1 = \epsilon_2$. Assuming that the amplitudes are the same (that is, $E_{0I} = E_{0R}$) the following equation will represent the resultant wave:
\begin{equation}
E(x,t) = 2E_{0I} \sin kx \cos \omega t
\end{equation}
The point of lowest intensity on a standing wave is called a node, and the point of highest amplitude is an amplitude. These are significant concepts in electromagnetic theory. 

\section{Addition of Waves of Different frequency}
\subsection{Beats}

Beats contain 2 waves at different frequencies traveling in the same direction. They have the same initial phase angles (can assume this to be zero): 
\[E_1 = E_{01} \cos (k_1x - \omega _1 t)\]
\[E_2 = E_{02} \cos (k_2x - \omega _2 t)\]
It can be shown that the total disturbance will take the form if the waves have equal amplitudes and zero initial phase angles:
\begin{equation}
E = 2E_{01} \cos (k_mx-\omega_m t) \cos (\bar{k}x-\bar{\omega}t)
\end{equation}
where
\[\bar{\omega} \equiv \frac{1}{2}(\omega_1+\omega_2)\]
\[\omega_m \equiv \frac{1}{2}(\omega_1-\omega_2)\]
\[\bar{k} \equiv \frac{1}{2}(k_1+k_2)\]
\[k_m \equiv \frac{1}{2}(k_1-k_2)\]
The difference terms are dominated by the low frequency component, whereas the average terms of \textomega {}  and k are dominated by the high frequency component. The low frequency wave profile essentially encompasses the high frequency wave. The low frequency wave moves at the group velocity, $v_g = \frac{\omega_m}{k_1} = \left( \frac{\partial \omega }{\partial k} \right)_{\bar{\omega}}$ whereas the high frequency carrier wave travels at a phase velocity $v_c = \frac{\bar{\omega}}{\bar{k}}$. $\omega(k)$ is the dispertion relation which is a property of the medium. 

\section{Anharmonic Periodic Waves}
Some waves are not harmonic. These are hard to analyzed, so they are instead represented as superpositioning harmonic waves so that they're continuous and differentiable. The Fourier series is commonly used for this. A periodic function $f(x)$ can be represented by the following series:
\begin{equation}
f(x) = \frac{A_0}{2}+ \displaystyle\sum\limits_{m=1}^\infty A_m \cos mkx + \displaystyle\sum\limits_{m=1}^\infty B_m \sin mkx
\end{equation}
where
\begin{equation}
A_0 = \frac{2}{\lambda} \int_0^\lambda f(x)dx
\end{equation}
\begin{equation}
A_m = \frac{2}{\lambda}\int_0^\lambda f(x) \cos mkx dx
\end{equation}
\begin{equation}
B_m = \frac{2}{\lambda}\int_0^\lambda f(x) \sin mkx dx
\end{equation}

This Fourier Series Analysis business can be extrapolated into two dimensions for the discrete Fourier transform which is commonly used in image processing. 
\section{Nonperiodic Waves}
Nonperiodic waves do not repeat continuously. This makes Fourier series somewhat of an awkward tool to use for analysis. The Fourier integral was designed to handle such signals which are often caused by pulses and wave packets. The governing equations of the Fourier integral are as follows:
\begin{equation}
f(x) = \frac{1}{\pi} \left[ \int _0^\infty A(k) \cos kx dk + \int_0^\infty B(k) \sin kxdk \right]
\end{equation}
where:
\begin{equation}
A(k) = \int_{-\infty}^\infty f(x) \cos kxdx
\end{equation}
\begin{equation}
B(k) = \int_{-\infty}^\infty f(x) \sin kxdx
\end{equation}

\section{Glossary}
\begin{description}
\item[Absorption band:] The wavelength band within which materials absorb electromagnetic energy.
\item[Anomalous dispersion:] When the group velocity is greater than the carrier velocity in a system of waves with multiple frequencies. 
\item[Coherence length:] The spatial length in which a wave remains within its frequency bandwidth. 
\item[Coherence time:] The time in which a wave remains in its allocated frequency bandwidth. 
\item[Coherent:] Waves are coherent when their initial phase difference $\varepsilon_1-\varepsilon_2 = a$ where $a$ is a constant.
\item[Constructive interference:] Occurs when interference among waves causes an overall increase in the intensity of the disturbance. 
\item[Destructive interference:] When the interference between waves causes an overall decrease in intensity of the distrubance. 
\item[Dispersive medium:] A medium in which the phase velocity of a wave or group of waves depends on its frequency. 
\item[Group velocity:] The velocity of some shape or leading edge of a pulse, it is taken as the rate at which a feature moves to be the velocity of the group of waves as a whole. 
\item[Normal dispersion:] When group velocity is less than the carrier velocity of waves with multiple frequencies. 
\item[Optical path difference (OPD):] The difference in two optical paths, $n(x_1-x_2)$
\item[Power spectrum:]  A measure of the distribution of energy, or power, at each and every component frequency that has been analyzed using a discrete Fourier transform.
\item[Standing or Stationary wave:] A wave whose profile does not move through space.
\item[Wave packet:] A small part of a wavetrain. One pulse of a wavetrain specifically that assemble together as a continuous range of spatial frequencies. 
\end{description}

\section{Useful Equations}
Sum of 2 waves: 
\begin{equation}
E = E_0 \sin (\omega t + \alpha)
\end{equation}
\begin{equation}
E_0 ^2 = E_{01} ^2 + E_{02} ^2 + 2E_{01}E_{02} \cos (\alpha_2 - \alpha_1)
\end{equation}
\begin{equation}
\tan \alpha = \frac{E_{01} \sin \alpha_1 + E_{02} \sin \alpha_2}{E_{01} \cos \alpha_1 + E_{02} \cos \alpha_2}
\end{equation}

Fourier Series:
\begin{equation}
f(x) = \frac{1}{\pi} \left[ \int _0^\infty A(k) \cos kx dk + \int_0^\infty B(k) \sin kxdk \right]
\end{equation}
where:
\begin{equation}
A(k) = \int_{-\infty}^\infty f(x) \cos kxdx
\end{equation}
\begin{equation}
B(k) = \int_{-\infty}^\infty f(x) \sin kxdx
\end{equation}

\chapter{Electromagnetic Theory, Photons, and Light}
\chapter{Polarization}
\section{The nature of polarized light}
Light in this text has previously been considered to be linearly polarized or plane-polarized. In this case, the orientation of the electrical field is constant, but it's magnitude as sign can vary in time. This disturbance occurs in a plane called a plane of vibration which is a fixed plane containing the electric field and propagation vectors. 

Light waves can also act in such a way that their electric field orientations are mutually perpendicular. The superposition of these waves may not be linearly polarized as they are when 2 linearly polarized waves interact. 

\subsection{Linear Polarization}
Two orthogonal optical waves can be represented as:
\[\vec{E}_x(z,t)=\hat{i}E_{0x} \cos (kz-\omega t)\]
\[\vec{E}_y(z,t)=\hat{j}E_{0y} \cos (kz-\omega t+\varepsilon)\]
where $\varepsilon$ is the relative phase difference between the waves which are both traveling in the z-direction. The resultant wave when the two are in phase is:
\[\vec{E}=(\hat{i}E_{0x}+\hat{j}E_{0y}) \cos (kz-\omega t)\]
This wave is linearly polarized on a plane that is some angle between the x-z and y-z planes. 

If $\varepsilon$ is an odd integer multiple of $\pi$ such that the waves are 180 degrees out of phase then the resulting equation becomes:
\[\vec{E}=(\hat{i}E_{0x}-\hat{j}E_{0y}) \cos (kz-\omega t)\] which is a rotated version of the previous equation. 

\subsection{Circular Polarization}
When both waves have the same amplitude, yet vary in phase by some form of $\pm \frac{\pi}{2}$, they are said to be circularly polarized. The two waves resemble:
\[\vec{E}_x(z,t)=\hat{i}E_{0x} \cos (kz-\omega t)\]
\[\vec{E}_x(z,t)=\hat{i}E_{0x} \sin (kz-\omega t)\]
and form a consequent waves:
\[\vec{E} = E_0 \left[\hat{i} \cos (kz-\omega t) +\hat{j}\sin (kz-\omega t) \right]\]
When light moving toward an observer is rotating clockwise, the wave is called right-circularly clockwise. Similarly, a wave rotating counter clockwise as it travels toward an observer is left-circularly polarized. In circularly polarized light, the endpoint travels in a circle around the axis that the wave is traveling at a rate of 1 rotation per wavelength. 
\subsection{Elliptical Polarization}
Both linear and circular polarized light are types of elliptical polarization (they're at the extremes of the light). This type of polarization is when the end location of light travels in an ellipse on a plane perpendicular to the propagation vector. This is the case when the equation describing the two waves are independent in of space or time (the kz - \textomega t is gone). The equations mentioned so far in this chapter can be altered to the form:
\begin{equation}
\left( \frac{E_y}{E_{0y}} \right) ^2 + \left( \frac{E_x}{E_{0x}} \right) ^2 -2 \left( \frac{E_x}{E_{0x}} \right) \left( \frac{E_y}{E_{0y}} \right) \cos \varepsilon = \sin ^2 \varepsilon
\end{equation}

Which is a plane making angle \textalpha with the $(E_x,E_y)$-coordinate system such that:
\begin{equation}
\tan 2\alpha = \frac{2E_{0x}E_{0y}\cos \varepsilon}{E_{0x}^2 - E_{0y}^2}
\end{equation}

The state of polarization can be linearly/planarly plarized in a $\mathcal{P}$-state, it can be right or left circularly polarized in an $\mathcal{R}$ or $\mathcal{L}$ state, and finally is can be in a state of elliptical polarization which is designated as an $\mathcal{E}$ state. 

\subsection{Natural light}
Light sources which emit random rapidly varying successions of different polarization states are referred to as natural light (also as unpolarized light). Light found in nature and man-made light are neither made completely of polarized or completely unpolarized lights. The electric field vector usually varies in what is known as being partially polarized. Do note that the infinite nature of monochromatic light causes it to be perfectly polarized.

\subsection{Angular Momentum and Photon Picture} 

\section{Polarizers}

\section{Glossary}
\begin{description}
\item[Analyzer: ]
\item[Birefringence: ]
\item[Brewster's angle: ]
\item[Circular birefringence: ]
\item[Circularly polarized: ] A result of multiple waves traveling to an observer that appear to rotate circularly on a plane perpendicular to the axis of propagation. 
\item[Cleavage plane: ]
\item[Compensator: ]
\item[Dichroism: ]
\item[Extraordinary ray: ]
\item[Faraday effect: ]
\item[Fast axis: ]
\item[Half-wave plate: ]
\item[Jones vector: ]
\item[Kerr effect: ]
\item[Liquid Crystal: ]
\item[Malus's law: ]
\item[Mueller matrix: ]
\item[Optic axis: ]
\item[Optical cavity: ]
\item[Ordinary ray: ]
\item[Polarization state: ] describes if light is in a: linear or planar $\mathcal{P}$-state; right or left circularly polarized in an $\mathcal{R}$ or $\mathcal{L}$ state; or in a state of elliptical polarization which is designated as an $\mathcal{E}$ state. 
\item[Retarder: ]
\item[Stress birefringence: ]
\item[Stokes parameters: ]
\item[Transmission axis: ]
\item[Twisted nematic cell: ]
\item[Uniaxial crystal: ]
\item[Unpolarized light: ]

\end{description}

\chapter{Interference}
\chapter{Modern Optics - Lasers and Such}



\part{Light and Matter}

\chapter{The Propagation of light}
\chapter{Diffraction}
\chapter{Basics of Coherence Theory}

\part{Analysis Techniques}
\chapter{Geometrical Optics}
\chapter{More on Geometrical Optics}
\chapter{Fourier Optics}


\end{document}
